% !TEX encoding = UTF-8
% !TEX TS-program = pdflatex
% !TEX root = ../tesi.tex

%**************************************************************
\chapter{Descrizione dello stage}
\label{cap:descrizione-stage}

%**************************************************************
\section{Il progetto}\label{sec:progetto}

Il progetto è nato dalla necessità dell'azienda \company{} di automatizzare il processo di configurazione del motore semantico di \emph{Engagent}\glsfirstoccur{}, il quale richiede la creazione manuale di \emph{synset}\glsfirstoccur{} e \emph{regole}\glsfirstoccur{}. Questo processo è economicamente fattibile e vantaggioso finché le dimensioni delle regole create sono ridotte, ma il costo cresce esponenzialmente con l'aumentare delle regole.\\
Più regole rendono più precisa la chatbot, ma incrementano di conseguenza i \emph{synset}\glsfirstoccur{} da inserire, prolungando i tempi di compilazione dell'\emph{NLP}\glsfirstoccur{} da qualche giorno a settimane.

%**************************************************************
\section{Obiettivi Aziendali}
L'obiettivo di automatizzare il processo di configurazione di Engagent non è nato con il progetto di stage, ma qualche anno fa, mentre il \textit{Machine Learning} diventava sempre più popolare. \company{} attribuì questo compito a un team esterno. Il loro compito consisteva nell'implementazione di un algoritmo di \textit{ML}, per la generazione di cluster contenenti gli ingredienti essenziali alla configurazione di Engagent.
Verso l'inizio dell'anno, i progressi fatti da questo team erano convincenti, quindi \company{} aveva l'intenzione di sperimentare l'integrazione di tali risultati con il proprio sistema.\\
Dallo stage, l'azienda si aspettava i seguenti prodotti:
\begin{itemize}
    \item \textbf{Studio di fattibilità:} analisi del problema\riferimento{cap:analisi-requisiti} e \textit{scouting}\glsfirstoccur{} di soluzioni già esistenti;
    \item \textbf{\app{}:} un'applicazione per la creazione del modello per il motore semantico di Engagent;
    \item \textbf{manuale:} manuale tecnico per l'utilizzo e la manutenzione dell'applicazione.    
\end{itemize}
%**************************************************************
\section{Obiettivi personali}

Durante la ricerca dell'azienda per lo stage, volevo contribuire a un progetto software in ambito professionale, facendo contemporaneamente i primi passi nel mondo delle intelligenze artificiali.\\
Il progetto proposto da PAT racchiudeva queste prerogative: sarei stato inserito in un progetto maturo e, con l'aiuto di esperti nel settore, avrei potuto lavorare con degli algoritmi di clustering.   

%**************************************************************
\section{Pianificazione}

Con l'aiuto del tutor aziendale, ho redatto il piano di lavoro, che comprende 312 ore distribuite in 8 ore al giorno, per 5 giorni alla settimana (lunedì 24 luglio mi sono dovuto assentare da lavoro, con il consenso del tutor aziendale, per un esame universitario).\\
La pianificazione ha avuto delle modifiche durante l'avanzare del progetto, vista la sua natura "sperimentale". Per esempio, il linguaggio di programmazione Python è stato accordato assieme al tutor aziendale solamente dopo un'analisi approfondita del problema.
Di seguito viene riportata l'ultima versione del piano di lavoro.
\begin{itemize}
    \item \textbf{I settimana:}
    \begin{itemize}
        \item studio della piattaforma \emph{Engagent};
    \end{itemize}
    \item \textbf{II settimana:}
    \begin{itemize}
        \item analisi e stesura di un report, riguardante il problema della creazione automatica del file di configurazione\riferimento{sec:progetto};
        \item preparazione dell'ambiente di lavoro;
    \end{itemize}
    \item \textbf{III settimana:}
        \begin{itemize}
            \item ricerca e sperimentazione di possibili soluzioni già esistenti per automatizzare la generazione di sinonimi;
            \item analisi e progettazione (al alto livello) di \app{};
            \item progettazione di dettaglio e implementazione del model\riferimento{sec:progettazione:model};
        \end{itemize}
    \item \textbf{IV settimana:}
    \begin{itemize}
        \item implementazione di \app{};
        \item stesura di test di unità;
    \end{itemize}
    \item \textbf{V settimana:}
    \begin{itemize}
        \item implementazione e miglioramento delle prestazioni di \emph{\app{}};
        \item verifica dei risultati di \emph{\app{}} da parte del tutor aziendale;
    \end{itemize}
    \item \textbf{VI settimana:}
    \begin{itemize}
        \item analisi per il miglioramento dei risultati di \emph{\app{}}
        \item progettazione di dettaglio e implementazione;
        \item documentazione;
    \end{itemize}
    \item \textbf{VII settimana:}
    \begin{itemize}
        \item documentazione e validazione;
    \end{itemize}
    \item \textbf{VIII settimana:}
    \begin{itemize}
        \item collaudo.
    \end{itemize}
\end{itemize}
