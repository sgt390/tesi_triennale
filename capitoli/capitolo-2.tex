% !TEX encoding = UTF-8
% !TEX TS-program = pdflatex
% !TEX root = ../tesi.tex

%**************************************************************
\chapter{Descrizione dello stage}
\label{cap:descrizione-stage}

%**************************************************************
\section{Il progetto}\label{sec:progetto}

Il progetto è nato dalla necessità dell'azienda \company di automatizzare il processo di configurazione del motore semantico di \emph{Engagent}\glsfirstoccur, il quale richiede la creazione manuale di \emph{synset}\glsfirstoccur e \emph{regole}\glsfirstoccur. Questo processo è economicamente fattibile e vantaggioso finché le dimensioni delle regole create sono ridotte, ma il costo cresce esponenzialmente con l'aumentare delle regole.\\
Più regole rendono più precisa la chatbot, ma incrementano di conseguenza i \glsfirstoccur{synset} da inserire, prolungando i tempi di compilazione del'\emph{NLP}\glsfirstoccur da qualche giorno a settimane.\\
Per risolvere questo problema, \company ha scomposto il processo nei seguenti \textit{task} da automatizzare:
\begin{enumerate}
    \item creazione delle \emph{regole}\glsfirstoccur;
    \item creazione dei synset;
    \item raffinamento dei risultati;
    \item creazione del file di configurazione \emph{NLP} per il motore semantico;
\end{enumerate}

\subsection{Creazione delle regole}
Questo task è il più difficile da automatizzare, perché richiede la definizione di \emph{match} contenenti categorie correlate tra loro. Inoltre, non esistono delle regole uguali per tutti, ma ogni settore ha regole diverse.\\
La soluzione è stata trovata nell'intelligenza artificiale, più in particolare nel clustering. Tramite l'analisi di \emph{chat} e \textit{FAQs} archiviate, è possibile generare delle regole allo stato grezzo.\\
Il problema è la bassa affidabilità dei risultati. Possibili soluzioni:
\begin{itemize}
    \item più dati in input (poco realizzabile nel breve periodo);
    \item algoritmo più complesso (in lavorazione);
    \item raffinamento manuale dei risultati (anche se manuale, richiede meno tempo di creare l'NLP da zero, soluzione migliore nel breve termine); 
    \item raffinamento automatico dei risultati (tramite \emph{POS-tagging}) (buon compromesso tra il raffinamento manuale e le altre due soluzioni).
\end{itemize}

\subsection{Creazione dei synset}
La creazione dei \emph{synset} è facilmente automatizzabile (se le regole sono già state create), in quanto basta trovare i sinonimi delle categorie.

\subsection{Raffinamento dei risultati}
I risultati dell'algoritmo di clustering devono essere ripuliti da \emph{stop-words} e regole prive di significato.

\subsection{creazione dell'NLP}
Adattamento dell'output dell'algoritmo di clustering al motore semantico di Engagent.

%**************************************************************
\section{Obbiettivi Aziendali}
L'obbiettivo di automatizzare il processo di configurazione di Engagent non è nato con il progetto di stage, ma qualche anno fa, mentre l'\textit{IA} diventava sempre più popolare. L'azienda attribuì questo compito a un team esterno. Il loro compito consisteva nel sviluppare un algoritmo di intelligenza artificiale, per la generazione di cluster contenenti gli ingredienti essenziali alla configurazione del motore semantico.\\
Verso l'inizio dell'anno, i progressi fatti da questo team erano convincenti, quindi \company aveva l'intenzione di sperimentare l'integrazione di tali risultati con il proprio sistema.

%**************************************************************
\section{Obbiettivi personali}

Durante la ricerca dell'azienda per lo stage, volevo contribuire a un progetto software in ambito professionale, facendo contemporaneamente i primi passi nel mondo delle intelligenze artificiali.\\
Il progetto proposto da PAT racchiudeva queste prerogative: sarei stato inserito in un progetto maturo e, con l'aiuto di esperti nel settore, avrei potuto lavorare con degli algoritmi di clustering.   

%**************************************************************
\section{Pianificazione}
Con l'aiuto del tutor aziendale ho redatto il piano di lavoro, che comprende 300 ore distribuite in 8 ore al giorno, per 5 giorni alla settimana.\\
La pianificazione ha avuto delle modifiche durante l'avanzare del progetto, vista la sua natura "sperimentale". Per esempio, il linguaggio di programmazione Python è stato accordato assieme al tutor aziendale solamente dopo un'analisi approfondita del problema.
Di seguito viene riportata l'ultima versione del piano di lavoro.
\begin{itemize}
    \item \textbf{I settimana:} studio della piattaforma \emph{Engagent};
    \item \textbf{II settimana:} analisi e stesura del report riguradante il problema descritto in \ref{sec:progetto};
    \item \textbf{III settimana:} ricerca e sperimentazione di possibili soluzioni già esistenti per automatizzare la generazione di sinonimi; analisi e progettazione dell' applicazione NLP-Generator;
    \item \textbf{IV settimana:} preparazione dell'ambiente di lavoro; codifica di NLP-Generator; stesura di test di unità;
    \item \textbf{V settimana:} codifica e miglioramento delle prestazioni di \emph{NLP-Generator}; verfica dei risultati di \emph{NLP-generator} da parte del tutor aziendale
    \item \textbf{VI settimana:} analisi sul miglioramento dei risultati di \emph{NLP-Generator} e codifica; documentazione;
    \item \textbf{VII settimana:} documentazione e validazione;
    \item \textbf{VIII settimana:} collaudo.
\end{itemize}

%**************************************************************
\section{Analisi dei rischi}

Assieme al tutor aziendale abbiamo individuato alcuni rischi a cui si potrà andare incontro. Per ognuno è stata trovata una soluzione:

\begin{risk}{Conflitti durante il testing e la modifica in Engagent}
    \riskdescription{La piattaforma Engagent mette a disposizione un ambiente di testing, per provare le configurazioni prima di portarle in produzione. Se due persone lavorano sullo stesso \emph{NLP} è possibile che si generino dei conflitti dannosi all'interno del software.}
    \risksolution{A ogni membro del team è stato assegnato un dominio su cui lavorare (un sotto-ambiente di testing) per evitare conflitti}
    \label{risk:conflict_engagent} 
\end{risk}
