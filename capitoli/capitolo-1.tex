% !TEX encoding = UTF-8
% !TEX TS-program = pdflatex
% !TEX root = ../tesi.tex

%**************************************************************
\chapter{Introduzione}
\label{cap:introduzione}
%**************************************************************
\section{L'azienda}

\company{} è un'azienda italiana che da 25 anni sviluppa soluzioni software per altre aziende e privati.\\
L'azienda lavora su 5 diversi macro-progetti, uno dei quali è Engagent, la chatbot professionale oggetto dei miei due mesi di stage.\\
Dal 2013, PAT è entrata a far parte di Zucchetti Group.
\begin{figure}[H]
    \centering
    \includegraphics[width=0.3\columnwidth]{logo-pat.png} 
    \caption{logo \company{}}
    \label{logo:pat}
\end{figure}

\subsection{Prodotti e servizi}
L'azienda offre ai suoi clienti l'automatizzazione dei processi e il miglioramento della \textit{user experience} dei loro prodotti.
PAT concretizza questi obiettivi attraverso i seguenti prodotti:
\begin{itemize}
    \item Engagent: chatbot semi-automatizzata per uso professionale, l'unico prodotto di \company{} con cui sono entrato in contatto; 
    \item Helpdesk: \gls{servicedeskg}\glsfirstoccur;
    \item Infinite: \textit{CRM} software orientato alla relazione tra cliente e azienda;
    \item Brain \textit{Interactive}: piattaforma per governare dei servizi personalizzati attraverso dei diagrammi di flusso; 
    \item Teammee: piattaforma per la comunicazione tra i dipendenti in un'azienda, usando la logica dei social networks;
\end{itemize}

\subsubsection{Engagent}

Engagent è una chatbot orientata al business, con un agente virtuale integrato. In \textit{backend}, un motore semantico permette di capire cosa sta chiedendo l'utente e trovare la risposta più coerente. Se la domanda è troppo complessa, il motore semantico estrae la categoria della domanda e la inoltra all'operatore adeguato.
\begin{figure}[H]
    \centering
    \includegraphics[width=0.3\columnwidth]{logo_engagent.png} 
    \caption{logo Engagent}
    \label{logo:engagent}
\end{figure}
\subsection{Organizzazione e Metodo di lavoro}
L'azienda è divisa in più gruppi di lavoro, uno per ogni macro-progetto, oltre alla segreteria e direzione.\\
Ogni team è separato dagli altri, anche se la collaborazione tra le parti è necessaria.\\
Tutti i team di sviluppo in \company{} seguono una metodologia Agile\footcite{site:agile-manifesto}. Questa fa parte delle metodologie iterative, caratterizzata da brevi iterazioni (o sprint, di circa 3-4 settimane) seguite dalla \textit{review} del lavoro svolto. Il focus principale si trova nel cliente: ci deve essere una interazione costante per capire quali sono le \textit{feature} più importanti, che hanno precedenza sulle altre.\\
Il team a cui ho preso parte applica questa metodologia. Il contatto con il cliente è frequente, che sia manutenzione o nuove \textit{features} da implementare. La piccola dimensione del team e le riunioni giornaliere permettono una buona collaborazione. Il team è gestito da un responsabile che organizza le riunioni e comunica con il manager dell'azienda.\\
Per quanto mi riguarda, ho adottato senza difficoltà queste metodologie, perché molto simili a quelle utilizzate durante il progetto di ingegneria del software.
\begin{figure}[H]
    \centering
    \includegraphics[width=0.3\columnwidth]{agile.png} 
    \caption{Agile}
    \label{logo:agile}
\end{figure}

%**************************************************************
\section{Strumenti e Tecnologie}
Questa sezione descrive, ad alto livello, le tecnologie utilizzate durante lo stage.
\subsection{Ambiente di lavoro}
L'ambiente di lavoro utilizzato è Windows 10, assieme al pacchetto office per la maggior parte delle attività.\\
I team di sviluppo hanno libera scelta sugli editor.\\Per la codifica e la stesura dei documenti, ho scelto Visual studio. Per la creazione di diagrammi UML, ho utilizzato Astah UML.\\
Ogni sviluppatore ha a disposizione un PC fisso e un portatile per le riunioni.

\subsection{Linguaggi di programmazione}
I linguaggi di programmazione che ho utilizzato sono:
\begin{itemize}
    \item \textbf{Python:} per la codifica dell'applicazione;
    \item \textbf{Shell:} per la creazione di script per automatizzare i seguenti task:
    \begin{itemize}
        \item esecuzione del programma tramite linea di comando;
        \item pulizia della cache del programma;
        \item pulizia dei file di output del programma.
    \end{itemize}
\end{itemize}

\subsection{Natural Language Processing}
In questo documento ho utilizzato termini riguardanti il \textit{natural language processing}, abbreviato \textit{NLP}.\\ 
L'NLP è derivato da più settori scientifici, in particolare l'intelligenza artificiale e la linguistica\footcite{site:nlp}. Si occupa dell'interazione tra i linguaggi umani e macchina, ed è usato, per esempio, per lo sviluppo di assistenti vocali\footcite{alexanlu}, correttori del testo, filtri antispam, motori di ricerca del testo\footcite{site:nlpexamples}.\\
Ho sfruttato l'NLP per eseguire i seguenti \textit{task}:
\begin{itemize}
    \item \textbf{lemmatizzazione:} trasformazione della parola nella sua forma base. Durante l'analisi di una frase, è necessario portare tutte le parole a una forma standard (esempio: giocavo -> giocare);
    \item \textbf{part-of-speech tagging o POS tagging:} analisi grammaticale del testo, attribuendo il giusto tipo a ogni parola (verbo, nome, aggettivo, pronome, articolo). Questo permette di filtrare solo le parti del testo desiderate (ad esempio, rimuovendo tutti gli articoli);
    \item \textbf{stemming:} estrazione della radice della parola. Come per la lemmatizzazione, permette di rendere ancora più generali i risultati, ma meno precisi. Grazie ai vari esperimenti eseguiti durante il progetto, ho capito che lo \textit{stemming} è particolarmente utile per l'inglese, ma poco efficace con parole italiane;
    \item \textbf{Parsing:} analisi logica del testo, per individuare soggetto, predicato e complemento. Questo processo ha l'obiettivo di ridurre una frase complessa in una semplice, molto più corta ma con lo stesso significato. Sarebbe il migliore dei \textit{task} ma è difficle trovare un buon parser, in quanto è molto complesso da implementare.
\end{itemize}

%**************************************************************
\section{Organizzazione del testo}

\begin{description}
    \item[{\hyperref[cap:introduzione]{Il primo capitolo}}] contiene una panoramica dell'azienda e le tecnologie utilizzate \company{};

    \item[{\hyperref[cap:descrizione-stage]{Il secondo capitolo}}] contiene la pianificazione progetto; 
    
    \item[{\hyperref[cap:analisi-requisiti]{Il terzo capitolo}}] approfondisce nel dettaglio i requisiti del progetto, descrivendo il processo di analisi che ha portato alla loro stesura;
    
    \item[{\hyperref[cap:progettazione-codifica]{Il quarto capitolo}}] approfondisce l'architettura del software sviluppato, con il supporto di esempi specifici e schemi ad alto livello;
        
    \item[{\hyperref[cap:conclusioni]{Il quinto capitolo}}] contiene il resoconto del progetto e considerazioni personali sullo stage. 
\end{description}