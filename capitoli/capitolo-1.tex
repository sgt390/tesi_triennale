% !TEX encoding = UTF-8
% !TEX TS-program = pdflatex
% !TEX root = ../tesi.tex

%**************************************************************
\chapter{Introduzione}
\label{cap:introduzione}
%**************************************************************
\section{L'azienda}

\company è un'azienda italiana che da 25 anni sviluppa soluzioni software per altre aziende e privati.\\
L'azienda lavora su 5 diversi macro-progetti, uno dei quali è Engagent, la chatbot professionale oggetto dei miei due mesi di stage.\\
Dal 2013, PAT è entrata a far parte di Zucchetti Group.

\subsection{Prodotti e servizi}
L'azienda offre ai suoi clienti l'automatizzazione dei processi e il miglioramento dell'\textit{user experience} dei loro prodotti.
PAT concretizza quesi obbiettivi attraverso i seguenti prodotti:
\begin{itemize}
    \item Engagent: chatbot semi-automatizzata per uso professionale; 
    \item Helpdesk;
    \item Infinite: \textit{CRM} software orientato alla relazione tra cliente e azienda;
    \item Brain \textit{Interactive}: piattaforma per governare dei servizi personalizzati attraverso dei diagrammi di flusso; 
    \item Teammee: piattaforma per la comunicazione tra i dipendenti in un'azienda, usando la logica dei social networks;
\end{itemize}

\subsubsection{Engagent}
Engagent è una chatbot orientata al business, con un agente virtuale integrato.\\
In \textit{backend}, un motore semantico permette di capire cosa sta chiedendo l'utente e trovare la risposta più coerente. Se la domanda è troppo complessa, il motore semantico estrate la categoria della domanda e la reinderizza all'operatore (umano) adeguato.\\

\subsection{Organizzazione e Metodo di lavoro}
L'azienda è divisa in più gruppi di lavoro, uno per ogni macro-progetto, oltre alla segreteria e direzione.\\
Ogni team è separato dagli altri, anche se la collaborazione tra le parti è molto presente.\\
Tutti i team di sviluppo in \company seguono una metodologia Agile. Questa metodologia fa parte delle metodologie iterative. Le brevi iterazioni (o sprint, di circa 3-4 settimane) sono seguite dalla \textit{review} del lavoro svolto. Il focus principale si trova nel cliente, difatti ci deve essere una interazione costante per capire quali sono le \textit{feature} più importanti, ovvero quelli da sviluppare prima.\\
Il team a cui ho preso parte segue correttamente questa metodologia: il contatto con il cliente è giornaliero, che sia manutenzione o nuove \textit{features} da sviluppare.
%**************************************************************
\section{Obbiettivi personali}

Volevo contribuire a un progetto software in ambito professionale, facendo contemporaneamente i primi passi nel mondo delle intelligenze artificiali.\\
Il progetto proposto da PAT racchiudeva queste prerogative: sarei stato inserito in un progetto maturo e, con l'aiuto di esperti nel settore, avrei potuto lavorare con un algoritmo di clustering.   

%**************************************************************
\section{Organizzazione del testo}

\begin{description}
    \item[{\hyperref[cap:processi-metodologie]{Il secondo capitolo}}] descrive i processi e le metodologie utilizzate durante lo stage;
    
    \item[{\hyperref[cap:descrizione-stage]{Il terzo capitolo}}] approfondisce i lati meno tecnici dello stage; 
    
    \item[{\hyperref[cap:analisi-requisiti]{Il quarto capitolo}}] approfondisce nel dettaglio i requisiti del progetto, descrivendo il processo di analisi che ha portato alla loro stesura;
    
    \item[{\hyperref[cap:progettazione-codifica]{Il quinto capitolo}}] approfondisce l'architettura del software sviluppato, con il supporto di esempi specifici e schemi ad alto livello;
    
    \item[{\hyperref[cap:verifica-validazione]{Il sesto capitolo}}] approfondisce le tecniche di verifica e validazione utilizzate durante lo stage, con i relativi risultati;
    
    \item[{\hyperref[cap:conclusioni]{Nel settimo capitolo}}] contiene il resoconto dello stage. 
\end{description}

Riguardo la stesura del testo, relativamente al documento sono state adottate le seguenti convenzioni tipografiche:
\begin{itemize}
	\item gli acronimi, le abbreviazioni e i termini ambigui o di uso non comune menzionati vengono definiti nel glossario, situato alla fine del presente documento;
	\item per la prima occorrenza dei termini riportati nel glossario viene utilizzata la seguente nomenclatura: \emph{parola}\glsfirstoccur;
	\item i termini in lingua straniera o facenti parti del gergo tecnico sono evidenziati con il carattere \emph{corsivo}.
\end{itemize}