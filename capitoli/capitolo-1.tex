% !TEX encoding = UTF-8
% !TEX TS-program = pdflatex
% !TEX root = ../tesi.tex

%**************************************************************
\chapter{Introduzione}
\label{cap:introduzione}
%**************************************************************
\section{L'azienda}

\company è un'azienda italiana che da 25 anni sviluppa soluzioni software per altre aziende e privati.\\
L'azienda lavora su 5 diversi macro-progetti, uno dei quali è Engagent, la chatbot professionale oggetto dei miei due mesi di stage.\\
Dal 2013, PAT è entrata a far parte di Zucchetti Group.

\subsection{Prodotti e servizi}
L'azienda offre ai suoi clienti l'automatizzazione dei processi e il miglioramento dell'\textit{user experience} dei loro prodotti.
PAT concretizza quesi obbiettivi attraverso i seguenti prodotti:
\begin{itemize}
    \item Engagent: chatbot semi-automatizzata per uso professionale; 
    \item Helpdesk;
    \item Infinite: \textit{CRM} software orientato alla relazione tra cliente e azienda;
    \item Brain \textit{Interactive}: piattaforma per governare dei servizi personalizzati attraverso dei diagrammi di flusso; 
    \item Teammee: piattaforma per la comunicazione tra i dipendenti in un'azienda, usando la logica dei social networks;
\end{itemize}

\subsubsection{Engagent}
Engagent è una chatbot orientata al business, con un agente virtuale integrato.\\
In \textit{backend}, un motore semantico permette di capire cosa sta chiedendo l'utente e trovare la risposta più coerente. Se la domanda è troppo complessa, il motore semantico estrate la categoria della domanda e la reinderizza all'operatore (umano) adeguato.\\

\subsection{Organizzazione e Metodo di lavoro}
L'azienda è divisa in più gruppi di lavoro, uno per ogni macro-progetto, oltre alla segreteria e direzione.\\
Ogni team è separato dagli altri, anche se la collaborazione tra le parti è necessaria.\\
Tutti i team di sviluppo in \company seguono una metodologia Agile. Questa metodologia fa parte delle metodologie iterative. Le brevi iterazioni (o sprint, di circa 3-4 settimane) sono seguite dalla \textit{review} del lavoro svolto. Il focus principale si trova nel cliente, difatti ci deve essere una interazione costante per capire quali sono le \textit{feature} più importanti, ovvero quelli da sviluppare prima.\\
Il team a cui ho preso parte applica questa metodologia. Il contatto con il cliente è frequente, che sia manutenzione o nuove \textit{features} da sviluppare. La piccola dimensione del team e le riunioni giornaliere permettono una buona collaborazione. Il team è gestito da un responsabile che organizza le riunioni e comunica con il manager dell'azienda.\\
Per quanto mi rigurarda, ho adottato senza difficoltà queste metodologie, perché molto simili a quelle utilizzate durante il progetto di ingegneria del software.

%**************************************************************
\section{Strumenti e Tecnologie}

\subsection{Ambiente di lavoro}
L'ambiente di lavoro utilizzato è Windows 10, assieme al pacchetto office per la maggior parte delle attività.\\
I team di sviluppo hanno libera scelta sugli editor. Ho scelto Visual studio, per la sua flessibilità.\\
Ogni sviluppatore ha a disposizione un PC fisso e un portatile per le riunioni.

\subsection{Test}
I test vengno eseguiti su più livelli:
\begin{itemize}
    \item \textbf{test di Engagent:} test di accettazione e di sistema. Viene verificato che il motore semantico funzioni correttamente;
    \item \textbf{Postman:} test di integrazione. Permette di generare delle richieste http alle API sviluppate;
    \item \textbf{pytest:} test di unità e integrazione. Ambiente di test specifico di Python. Tramite pytest-cov, permette di calcolare il code coverage;
    \item \textbf{pylint:} analisi statica del codice. Test statici per Python;
\end{itemize}

%**************************************************************
\section{Organizzazione del testo}

\begin{description}
    \item[{\hyperref[cap:introduzione]{Il primo capitolo}}] contiene una panoramica dell' azienda e le tecnologie utilizzate \company;

    \item[{\hyperref[cap:descrizione-stage]{Il secondo capitolo}}] contiene la pianificazione progetto; 
    
    \item[{\hyperref[cap:analisi-requisiti]{Il terzo capitolo}}] approfondisce nel dettaglio i requisiti del progetto, descrivendo il processo di analisi che ha portato alla loro stesura;
    
    \item[{\hyperref[cap:progettazione-codifica]{Il quarto capitolo}}] approfondisce l'architettura del software sviluppato, con il supporto di esempi specifici e schemi ad alto livello;
    
    \item[{\hyperref[cap:verifica-validazione]{Il quinto capitolo}}] approfondisce le tecniche di verifica e validazione utilizzate durante lo stage, con i relativi risultati;
    
    \item[{\hyperref[cap:conclusioni]{Nel sesto capitolo}}] contiene il resoconto dello stage. 
\end{description}

Riguardo la stesura del testo, relativamente al documento sono state adottate le seguenti convenzioni tipografiche:
\begin{itemize}
	\item termini di uso non comune vengono definiti nel glossario, situato alla fine del documento;
	\item per la prima occorrenza dei termini riportati nel glossario viene utilizzata la seguente nomenclatura: \emph{parola}\glsfirstoccur;
	\item i termini in lingua straniera o facenti parti del gergo tecnico sono evidenziati con il carattere \emph{corsivo}.
\end{itemize}